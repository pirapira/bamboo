\documentclass{book}

\usepackage{stmaryrd}
\usepackage{amsmath}


\newcommand{\todo}[1]{\underline{#1}}
\newcommand{\sem}[1]{\llbracket{#1}\rrbracket}
\newcommand{\evalE}[1]{E_\mathrm{e}\left({#1}\right)}
\newcommand{\evalS}[1]{E_\mathrm{s}\left({#1}\right)}

\newcommand{\expressionsentence}[1]{\mathsf{void}={#1}\mathsf{;}}

\begin{document}

\chapter{Syntax}

\section{Identifier}

\todo{define identifier}

Excercise: which of the following are identifiers? \\
\todo{complete}

\section{Syntactic Types}

Bamboo has some \textit{syntactic types}.
\begin{itemize}
\item \texttt{void}
\item \texttt{uint256}
\item \texttt{bool}
\end{itemize}
\todo{complete the list}.

Every syntactic type except \texttt{void} is a \textit{non-\texttt{void} syntactic type}.

\section{Expressions}

\section{Sentences}

\section{Blocks}

\section{Contracts}

\subsection{Contract's Signature}

$(\mathtt{auction}, [\mathtt{address}, \mathtt{uint256}, \mathtt{address}, \mathtt{uint256}])$.

Excercise: what is the signature of the following contract?
\todo{complete the question}

\chapter{Semantics}

\section{Notations}

\todo{describe $\in$}

\todo{describe pi}

\section{States}

\subsection{Interpretation of Syntactic Types}
Each syntactic type~$T$ is associated with a set~$\sem{T}$.

Excercise: determine if each of the following is true.
\begin{enumerate}
  \item $0 \in\sem{\mathtt{address}}$?
\end{enumerate}

A \textit{value} is a union of all $\sem{T}$ for $T \in \{\ldots\}$.

\subsection{A Contract's States}
A contract has \textit{states}.

When a contract has a signature $(x, [T_1, T_2, \ldots, T_n])$ ($n \ge 0$),
the set of the states of the contract is
$\Pi_0^{n} \sem{T_i}$.

\subsection{A Program's Account States}
A program determines a set of \textit{account states}.

\section{Dynamics}

\subsection{Variable Environment}

A \textit{variable environment} is a partial map that takes identifiers and may or may not return a value.  When a variable environemtent $\sigma$ maps and identifier~$i$ to a value~$v$, we write
\[
\sigma(i) = v
\]

\subsection{Current Call}

A \textit{currrent call} $c = (c_\mathrm{s}, c_\mathrm{v}, c_\mathrm{t})$ is a tuple of
\begin{itemize}
\item a value $c_\mathrm{s}$ called the \textit{sender},
\item a value $c_\mathrm{v}$ called the \textit{transferred ammount} and
\item a value $c_\mathrm{t}$ called the \textit{timestamp}.
\end{itemize}

\subsection{World Oracle}

\subsubsection{Call Queries}

\subsubsection{Create Queries}

\subsubsection{Balance Queries}

\subsubsection{World Oracle}

Both call queries and create queries are \textit{oracle queries}.

A \textit{world oracle} is defined coinductively as a function that takes an oracle query and returns a pair of a value and a world oracle.

When a world oracle~$w$ takes a query~$q$ and returns a pair of a value~$v$ and a world oracle~$w'$, we write $w(q) = (v, w')$.

\todo{Add a possibility that the world oracle calls into the program again.}

\subsection{Evaluation of an Expression}

The evaluation for expressions takes
\begin{itemize}
\item an expression,
\item a current call,
\item a world oracle and
\item a variable environment.
\end{itemize}
It returns
\begin{itemize}
\item a value and
\item a world oracle
\end{itemize}

\subsubsection{Evaluation of Literals}

true

false

now

this

msg.sender

msg.value

balance(x) sends a balance query on the world oracle.

\subsubsection{Evaluation of an Identifier}

\subsubsection{Evaluation of a new-expression}

\todo{Consider new-expressions with nontrivial continuation later.  That requires an interaction between the program and the world oracle. }

\subsubsection{Evaluation of a call-expression}

\todo{Consider call-expressions with nontrivial continuation later.  That requires an interaction between the program and the world oracle. }

\subsection{Evaluation of a Sentence}

The evaluation function for sentences take
\begin{itemize}
\item a sentence,
\item a current call,
\item a variable environment and
\item a world oracle
\end{itemize}
and returns
\begin{itemize}
\item a variable environment and
\item a world oracle.
\end{itemize}

\subsubsection{Evaluation of an Expression Sentence}

\[
\evalS{\boxed{\expressionsentence{e}}, c, \sigma, w} := (\sigma', w')
\]
where
\[
\evalE{\boxed{\expressionsentence{e}}, c, \sigma, w} = (v, \sigma', w')
\]

\end{document}
